	%%%%%%%%%%%%%%%%%%%%%%%%%%%%%%%%%%%%%%%%%
% Beamer Presentation
% LaTeX Template
% Version 1.0 (10/11/12)
%
% This template has been downloaded from:
% http://www.LaTeXTemplates.com
%
% License:
% CC BY-NC-SA 3.0 (http://creativecommons.org/licenses/by-nc-sa/3.0/)
%
%%%%%%%%%%%%%%%%%%%%%%%%%%%%%%%%%%%%%%%%%

%----------------------------------------------------------------------------------------
%	PACKAGES AND THEMES
%----------------------------------------------------------------------------------------

\documentclass{beamer}

\mode<presentation> {

% The Beamer class comes with a number of default slide themes
% which change the colors and layouts of slides. Below this is a list
% of all the themes, uncomment each in turn to see what they look like.
\usetheme[progressbar=frametitle]{metropolis}
\usepackage{appendixnumberbeamer}


%\setbeamertemplate{footline} % To remove the footer line in all slides uncomment this line
%\setbeamertemplate{footline}[page number] % To replace the footer line in all slides with a simple slide count uncomment this line

%\setbeamertemplate{navigation symbols}{} % To remove the navigation symbols from the bottom of all slides uncomment this line
}

\usepackage{graphicx} % Allows including images
\usepackage{booktabs} % Allows the use of \toprule, \midrule and \bottomrule in tables106
\usepackage{algorithm}
\usepackage{algpseudocode}
\usepackage{anyfontsize}
% \usepackage{lmodern}
\usepackage{mathtools}

\DeclareMathOperator{\spn}{span}
\DeclarePairedDelimiter{\floor}{\lfloor}{\rfloor}
\newcommand{\matr}[1]{\mathbf{#1}}


\usepackage{commath}



%----------------------------------------------------------------------------------------
%	TITLE PAGE
%----------------------------------------------------------------------------------------
\title{Git with the program}

\date{\today}
\author{Shuai Jiang}
\institute{Brown University}
% \titlegraphic{\hfill\includegraphics[height=1.5cm]{logo.pdf}}


\begin{document}

%----------------------------------------------------------------------------------------
%	PRESENTATION SLIDES
%----------------------------------------------------------------------------------------


\maketitle

\section{Introduction to Git}
\begin{frame}
	\frametitle{Why should I care?}

	\pause

	Probably the best question one can ask in grad school. 
	Some scenarios: \pause 
	\begin{enumerate}
		\item Graduate/drop-out and work in industry; probably involves coding because pure math is (practically) useless; \pause probably needs git. \pause
		\item Graduate and work in government lab; probably involves coding; \pause probably needs git.\pause
		\item Graduate and work in academia; might have to collaborate on code (?); \pause small probability needs git.\pause
		\item Graduate and be a bum; do not need git. \pause 
	\end{enumerate}
	Think of it like learning \LaTeX: initial time investment for rewards later on. 
\end{frame}

\begin{frame}
	\frametitle{What is Git?}
	\begin{itemize}
		\item Version control - ``system that records changes to a file or set of files over time so that you can recall specific versions later.'' \pause 
		\item For us graduate students in APMA, it's a way of backing up (and keeping track of) three important things:  .tex files,  code,  and data. \pause
	\end{itemize}
	\textbf{``BUT MARSHALL YOU DUMMY, THERE IS DROPBOX FOR THIS STUFF!'}
\end{frame}

\begin{frame}
	\frametitle{Scenarios where Dropbox fails.}
	\begin{enumerate}
		\item ``Last week when I ran this code, it produced linear convergence, but now it only produces sub-linear stuff'' - Git stores snapshots of the entire project, and you can revert to it! You have to revert individual files in Dropbox. \pause
		\item ``I wonder if this crazy idea will work... but I don't want to destroy what I had'' - You can create a \emph{branch} in Git which allows you to easily prototype, and allows you to merge your crazy idea. No corresponding feature in Dropbox. 
		\item ``Ugh, I forgot what I did a week ago with this code!'' - You can label each snapshot in Git with a message saying what you did. 
	\end{enumerate}
\end{frame}
\end{document}